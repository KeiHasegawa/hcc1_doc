\label{stmt_e006}
In this chapter, we'll discuss about the way of generating 3 address
codes for each statement. In chapter \ref{expr_e000}, we defined the
interface for expression as the class `{\tt{expr}}' and represented
the syntax analisis tree with it while bottom-up parsing. Samely,
we'll define the interface for statemnet like bellow
\begin{verbatim}
struct stmt {
  virtual void gen() = 0;  // Generate 3 address codes.
  virtual ~stmt(){}
};
\end{verbatim}
When the grammer symbol `{\tt{function-definition}}' is reduced,
frontend generates 3 address codes like bellow.
\begin{verbatim}
%{
/* yacc program text */
vector<_3ac*> code;

struct comp_stmt : stmt {
  ...
  void gen();  // Override pure virtual function
};
%}

%%

function_definition
  : function_header compound_statement
    { function_definition($2); }
  ;
...

%%

void function_definition(comp_stmt* cs)
{
  cs->gen();  // 3 address codes of this function are stored
              // in global container `code'.
  ...
}
\end{verbatim}

\section{{\tt{labeled-statement}}}

\subsection{label statement}
\label{stmt_e000}
The syntax analisis tree can be represented like bellow.
\begin{verbatim}
struct label_stmt : stmt {
  string m_label;
  stmt* m_stmt;
  ...
  void gen();
};
\end{verbatim}
Thinking about the situation where
a label definition is followed by {\tt{goto}} statement or
{\tt{goto}} statement is followed by a label definition,
we'll define two tables like bellow.
\begin{verbatim}
struct define {
  ...
  to3ac* m_to;
};

map<string, define> defined;  // Defined label

struct use {
  ...
  goto3ac* m_goto;
};

map<string, use> used;  // Used label without definition
\end{verbatim}
{\tt{label\_stmt::gen}} referencing these tables
becomes like bellow.
\begin{verbatim}
void label_stmt::gen()
{
  map<string, define>::const_iterator p = defined.find(m_label);
  if ( p != defined.end() ) {
    // Already defined. Error.
  }

  to3ac* to = new to3ac;
  code.push_back(to);
  defined[m_label].m_to = to;  // Add this label into table.

  map<string, vector<use> >::iterator p = used.find(m_label);
  if ( p != used.end() ) {
    // For each `goto3ac', decide `goto3ac::m_to'.
    const vector<use>& v = p->second;
    for_each(v.begin(),v.end(),bind2nd(ptr_fun(decide),to));
    used.erase(p);
  }
  m_stmt->gen();
}
\end{verbatim}
Please reference \ref{stmt_e001}.

\subsection{{\tt{case}} statement}
\label{stmt_e002}
The way of generating 3 address codes for {\tt{case}} statement
depends on the way of that for {\tt{switch}} statement.
Here we'll chose the way which is described at ``8.5 Switch statement
(Figure 8.27)'' of bibliography \cite{doragon}.
But please note that the way desrbied at bibliography \cite{doragon}
assumes that
{\tt{break}} statement of C language is executed after {\tt{case}} statement,
and that {\tt{default}} statement can only occur
after all {\tt{case}} statements.

The syntax analisis tree can be represented like bellow.
\begin{verbatim}
struct case_stmt : stmt {
  expr* m_expr;
  stmt* m_stmt;
  ...
  void gen();
};
\end{verbatim}
Becase {\tt{case}} statement shall occur inside {\tt{switch}} statement,
we'll define stack which contains {\tt{switch}} statement infomation.
\begin{verbatim}
struct case_info {
  ...
  var* m_expr;
  to3ac* m_to;
};

struct switch_info {
  ...
  vector<case_info> m_cases;
};

vector<switch_info*> switch_stack;

void case_stmt::gen()
{
  var* expr = m_expr->gen();
  expr = expr->rvalue();
  if ( !expr->isconstant() ) {
    // Not constant. Error.
  }
  const type* T = expr->m_type;
  if ( !T->integer() ) {
    // Not integer. Error.
  }
  if ( switch_stack.empty() ) {
    // Not inside switch statement. Error.
  }
  switch_info* top = switch_stack.top();
  vector<case_info>& v = top->m_cases;
  vector<case_info>::iterator p =
    find_if(v.begin(),v.end(),bind2nd(ptr_fun(cmp),expr));
  if ( p != v.end() ) {
    // There was the same case-value. Error.
  }
  to3ac* to = new to3ac;
  code.push_back(to);
  v.push_back(case_info(expr,to,...));
  return m_stmt->gen();
}
\end{verbatim}

\subsection{{\tt{default}} statement}
\label{stmt_e003}
The syntax analisis tree can be represented like bellow.
\begin{verbatim}
struct default_stmt : stmt {
  stmt* m_stmt;
  ...
  void gen();
};
\end{verbatim}
Samely with {\tt{case}} statement described at \ref{stmt_e002},
{\tt{default}} statement shall occur inside {\tt{switch}}.
For this, frontend references the stack used at \ref{stmt_e002} which contains
{\tt{switch}} statement infomation.
\begin{verbatim}
struct default_info {
  ...
  to3ac* m_to; 
};

struct switch_info {
  ...
  default_info m_default;
};

void default_stmt::gen()
{
  if ( switch_stack.empty() ) {
    // Not inside switch statement. Error.
  }

  switch_info* top = switch_stack.top();
  if ( top->m_default.m_to ) {
    // Already default statement existed in this
    // switch statement. Error.
  }

  to3ac* to = new to3ac;
  code.push_back(to);
  top->m_default.m_to = to;

  return m_stmt->gen();
}
\end{verbatim}

\section{{\tt{compound-statement}}}

In the grammer of bibliography \cite{ISO}, a declaration and a statement
can mix each other. The initializer code for non-static variable must
be executed in order. i.e. It must follows the initializer codes and
statement codes which occur before it and must be followed by
the initializer codes and
statement codes which occur after it.

The syntax analisis tree can be represented like bellow.
\begin{verbatim}
struct comp_stmt : stmt {
  typedef pair<vector<tac*>, stmt*> item;
  vector<item> m_items;
  scope* m_scope;
  ...
  void gen();
};
\end{verbatim}
{\tt{item::first}} is according to the initializer code.
{\tt{comp\_stmt::gen}} becomes like bellow.
\begin{verbatim}
void subr(const item& i)
{
  if ( stmt* s = i.second )
    s->gen();
  else {
    vector<tac*>& v = i.first;
    copy(v.begin(),v.end(),back_inserter(code));
  }
}

void comp_stmt::gen()
{
  struct x {
    scope* m_org;
     x(){ m_org = scope::current; }
    ~x(){ scope::current = m_org;  }
  } x;
  scope::current = m_scope;
  for_each(m_items.begin(),m_items.end(),subr);
}
\end{verbatim}

\section{{\tt{exrepssion-statement}}}

The syntax analisis tree can be represented like bellow.
\begin{verbatim}
struct expr_stmt : stmt {
  expr* m_expr;
  ...
  void gen()
  {
    var* expr = m_expr->eval();
    expr = expr->rvalue();
    const type* T = expr->m_type;
    if ( T is incomplete tagged type ) {
      // Errror.
    }
  }
};
\end{verbatim}

\section{{\tt{selection-statement}}}

\subsection{{\tt{if}} statement}

The syntax analisis tree can be represented like bellow.
\begin{verbatim}
struct if_stmt : stmt {
  expr* m_expr;
  stmt* m_stmt1;
  stmt* m_stmt2;  // statment following `else'
  ...
  void gen()
  {
    var* expr = m_expr->gen();
    expr->if_code(this);  // Call virtual function
  }
};
\end{verbatim}
As we mentioned at \ref{expr_e016}, the way of generating 3 address
codes for {\tt{if}} statement depends on the result of evaluating 
expression. {\tt{var::if\_code}} which generates most general codes
becomes like bellow.
\begin{verbatim}
void var::if_code(if_stmt* info)
{
  var* expr = rvalue();
  const type* T = expr->m_type;
  if ( !T->scalar() ) {
    // Not scalar. Error.
  }
  T = T->promotion();
  expr = expr->cast(T);
  var* zero = integer::create(0);
  zero = zero->cast(T);

  // Generate codes like if expression equals zero then jump.
  goto3ac* goto1 = new goto3ac(goto3ac::EQ,expr,zero);
  code.push_back(goto1);
  info->m_stmt1->gen();
  if ( !info->m_stmt2 ) {
    // No else case
    to3ac* to1 = new to3ac;
    code.push_back(to1);
    to1->m_goto.push_back(goto1);
    goto1->m_to = to1;
    return;
  }

  // Else case
  goto3ac* goto2 = new goto3ac;
  code.push_back(goto2);
  to3ac* to1 = new to3ac;
  code.push_back(to1);
  to1->m_goto.push_back(goto1);
  goto1->m_to = to1;
  info->m_stmt2->gen();
  to3ac* to2 = new to3ac;
  code.push_back(to2);
  to2->m_goto.push_back(goto2);
  goto2->m_to = to2;
}
\end{verbatim}
{\tt{var01}} in figure \ref{expr_e013} becomes 0 or 1 in
execution-time. We can use the characteristic.
The way of generating {\tt{if}} statement 3 address code for {\tt{var01}} 
becomes like bellow.
\begin{verbatim}
void var01::if_code(if_stmt* info)  // Override virtual function
{
  // Replace the codes like bellow
  //
  //   if y op z goto label
  //   res := 1             # This position is recorded in `m_one'
  //   goto end
  // label:
  //   res := 0             # This position is recorded in `m_zero'
  // end:
  //
  // to like bellow
  //
  //   if y op z goto label
  //   (Codes of info->m_stmt1)
  //   goto end
  // label:
  //   (Codes of info->m_stmt2)
  // end:
}
\end{verbatim}
For arithmetic constant and constant pointer,
the way of generating {\tt{if}} statement 3 address code
becomes like bellow.
\begin{verbatim}
template<class T> void common(constant<T>* ptr, if_stmt* info);

// Override virtual function
void constant<char>::if_code(if_stmt* info){ common(this,info); }
...
void constant<long double>::if_code(if_stmt* info){ common(this,info); }
void constant<void*>::if_code(if_stmt* info){ common(this,info); }

template<class T> void common(constant<T>* ptr, if_stmt* info)
{
  if ( ptr->zero() ) {
    int n = code.size();
    info->m_stmt1->gen();  // Once, gnerate 3 address code
    code.resize(n);        // and discard them
    if ( info->m_stmt2 )
      info->m_stmt2->gen();
  }
  else {
    info->m_stmt1->gen();
    if ( info->m_stmt2 ) {
      int n = code.size();
      info->m_stmt2->gen();  // Once, gnerate 3 address code
      code.resize(n);        // and discard them
    }
  }
}
\end{verbatim}
Here, frontend must check error for statements.
So, once generates 3 address codes and discards them.

\subsection{{\tt{switch}} statement}
\label{stmt_e005}
We mentioned about the way of generating 3 address codes for
``{\tt{case}} statement'' at \ref{stmt_e002} and
for ``{\tt{default}} statement'' at \ref{stmt_e003},
there, frontend just generated 3 address code `{\tt{to3ac}}' for
jump destination address and recorded the `{\tt{to3ac}}' in
outside {\tt{switch}} statement, especially, for {\tt{case}}
statement, also recorded the result of evaluating expression
in outside {\tt{switch}} statement.

The syntax analisis tree can be represented like bellow.
\begin{verbatim}
struct break_outer : vector<goto3ac*> {};

struct switch_stmt : stmt, break_outer {
  expr* m_expr;
  stmt* m_stmt;
  vector<case_info> m_cases;
  default_info m_default;
  ...
  void gen();
};
\end{verbatim}
The reasone why {\tt{switch\_stmt}} is derived from {\tt{break\_outer}}
is to update jump destination address according to {\tt{break}} statement.
\begin{verbatim}
stack<break_outer*> break_stack;

void switch_stmt::gen()
{
  var* expr = m_expr->gen();
  expr = expr->rvalue();
  const type* T = expr->m_type;
  if ( !T->integer() ) {
    // Not integer. Error.
  }
  goto3ac* start = new goto3ac;
  code.push_back(start);
  switch_stack.push(this);
  break_stack.push(this);
  m_stmt->gen();
  break_stack.pop();
  switch_stack.pop();
  goto3ac* last = new goto3ac;
  code.push_back(last);
  to3ac* to1 = new to3ac;
  code.push_back(to1);
  to1->m_goto.push_back(start);
  start->m_to = to1;

  // For each case statement, generate conditional jump 3 address code
  for_each(m_cases.begin(),m_cases.end(),bind1st(ptr_fun(gencode),expr));

  if ( to3ac* to = m_default.m_to ) {
    // There is default statement in this switch statement.
    // So, if condition doesn't hold for above case statement,
    // jump to default.
    goto3ac* go = new goto3ac;
    code.push_back(go);
    to->m_goto.push_back(go);
    go->m_to = to;
  }

  to3ac* to2 = new to3ac;
  code.push_back(to2);

  // Update destination address according to break statement
  for_each(begin(),end(),bind2nd(ptr_fun(update),to2));
  copy(begin(),end(),back_inserter(to2->m_goto));

  to2->m_goto.push_back(last);
  last->m_to = to2;
}
\end{verbatim}
For example,
\begin{verbatim}
void f(int n)
{
  switch ( n ) {
  case 0: stmt0
  case 2: stmt2
  default: stmtd 
  case 1: stmt1
  }
}
\end{verbatim}
3 address codes for above function becomes like bellow.
\begin{verbatim}
f:
  goto to1  # goto3ac* `start' of switch_stmt::gen
label0:
  (Code of stmt0)
label2:
  (Code of stmt2)
labeld:
  (Code of stmtd)
label1:
  (Code of stmt1)
  goto to2  # goto3ac* `last' of switch_stmt::gen
to1:
  if n == 0 goto label0
  if n == 2 goto label2
  if n == 1 goto label1
  goto labeld
to2:  # break statement destination address in switch statement
\end{verbatim}

\section{{\tt{iteration-statement}}}

\subsection{{\tt{while}} statement}
\label{stmt_e004}
The syntax analisis tree can be represented like bellow.
\begin{verbatim}
struct continue_outer : vector<goto3ac*> {};

struct while_stmt : stmt, break_outer, continue_outer {
  expr* m_expr;
  stmt* m_stmt;
  ...
  void gen()
  {
    to3ac* begin = new to3ac;
    code.push_back(begin);
    var* expr = m_expr->eval();
    expr->while_code(this,begin);  // Call virtual function
  }
};
\end{verbatim}
The reasone why {\tt{while\_stmt}} is derived from {\tt{break\_outer}}
is to update jump destination address according to {\tt{break}}
statement, samely with {\tt{switch\_stmt}}.
And more, the reason why
{\tt{while\_stmt}} is derived from {\tt{continue\_outer}}
is to update jump destination address according to {\tt{continue}}
statement.

Samely with {\tt{if}} statement code generation,
we'll call virtual function for the result of evaluating
expression. 
{\tt{var::while\_code}} which generates most general codes
becomes like bellow.
\begin{verbatim}
stack<continue_outer*> continue_stack;

void var::while_code(while_stmt* info, to3ac* begin)
{
  var* expr = rvalue();
  expr = expr->promotion();
  const type* T = expr->m_type;
  if ( !T->scalar() ) {
    // Not scalar. Error.
  }
  var* zero = integer::create(0);
  zero = zero->cast(T);

  // Generate codes like if expression equals to zero then leave loop.
  goto3ac* goto2 = new goto3ac(goto3ac::EQ,expr,zero);
  code.push_back(goto2);

  break_stack.push(info);
  continue_stack.push(info);
  info->m_stmt->gen();
  continue_stack.pop();
  break_stack.pop();

  // Update destination address according to continue statement
  continue_outer& c = *info;
  for_each(c.begin(),c.end(),bind2nd(ptr_fun(update),begin));
  copy(c.begin(),c.end(),back_inserter(begin->m_goto));

  // Jump to the loop head
  goto3ac* goto1 = new goto3ac;
  code.push_back(goto1);
  begin->m_goto.push_back(goto1);
  goto1->m_to = begin;

  to3ac* end = new to3ac;
  code.push_back(end);
  end->m_goto.push_back(goto2);
  goto2->m_to = end;

  // Update destination address according to break statement
  break_stmt::outer& b = *info;
  for_each(b.begin(),b.end(),bind2nd(ptr_fun(update),end));
  copy(b.begin(),b.end(),back_inserter(end->m_goto));
}
\end{verbatim}
For {\tt{var01}} in figure \ref{expr_e013},
the way of generating {\tt{while}} statement 3 address code
becomes like bellow.
\begin{verbatim}
// Override virtual function
void var01::while_code(while_stmt* info, to3ac* begin)
{
  // Replace the codes like bellow
  //
  //   if y op z goto label
  //   res := 1             # This position is recorded in `m_one'
  //   goto end
  // label:
  //   res := 0             # This position is recorded in `m_zero'
  // end:
  //
  // to like bellow
  //
  // begin:
  //   if y op z goto label
  //   (Code of info->m_stmt)
  //   goto begin
  // label:
}
\end{verbatim}
For arithmetic constant and constant pointer,
the way of generating {\tt{while}} statement 3 address code
becomes like bellow.
\begin{verbatim}
template<class T> void
common(constant<T>* ptr, while_stmt* info, to3ac* begin);

// Override virtual function
void constant<char>::while_code(while_stmt* info, to3ac* begin)
{ common(this,info,begin); }
...
void constant<long double>::while_code(while_stmt* info, to3ac* begin)
{ common(this,info,begin); }
void constant<void*>::while_code(while_stmt* info, to3ac* begin)
{ common(this,info,begin); }

template<class T> void
common(constant<T>* ptr, while_stmt* info, to3ac* begin)
{
  if ( !ptr->zero() ) {
    // Samely with var::while_code, except for entering
    // loop unconditionally.
  }
  else {
    int n = code.size();
    info->m_stmt->gen();  // Once, gnerate 3 address code
    code.resize(n);       // and discard them
  }
}
\end{verbatim}

\subsection{{\tt{do-while}} statement}

The syntax analisis tree can be represented like bellow.
\begin{verbatim}
struct do_stmt : stmt, break_outer, continue_outer {
  stmt* m_stmt;
  expr* m_expr;
  ...
  void gen()
  {
    to3ac* begin = new to3ac;
    code.push_back(begin);
    break_stack.push(this);
    continue_stack.push(this);
    m_stmt->gen();
    continue_stack.pop();
    break_stack.pop();
    to3ac* to2 = new to3ac;
    code.push_back(to2);
    continue_stmt::outer& c = *this;

    // Update destination address according to continue statement
    for_each(c.begin(),c.end(),bind2nd(ptr_fun(update),to2));
    copy(c.begin(),c.end(),back_inserter(to2->m_goto));

    var* expr = m_expr->gen();
    expr->do_code(this,begin);  // Call virtual function
                                // for generating the rest code.
  }
};
\end{verbatim}
Samely with {\tt{while}} statement, {\tt{do\_stmt}} is
derived from {\tt{break\_outer}} and {\tt{continue\_outer}}.
{\tt{var::do\_code}} which generates most general codes
becomes like bellow.
\begin{verbatim}
void var::do_code(do_stmt* info, to3ac* begin)
{
  var* expr = rvalue();
  expr = expr->promotion();
  const type* T = expr->m_type;
  if ( !T->scalar() ) {
    // Not scalar. Error.
  }
  var* zero = integer::create(0);
  zero = zero->cast(T);

  // Generate code like if expression doesn't equals to zero then
  // jump to the loop head.
  goto3ac* go = new goto3ac(goto3ac::NE,expr,zero);
  code.push_back(go);
  begin->m_goto.push_back(go);
  go->m_to = begin;

  to3ac* end = new to3ac;
  code.push_back(end);

  // Update destination address according to break statement
  break_stmt::outer& b = *info;
  for_each(b.begin(),b.end(),bind2nd(ptr_fun(update),end));
  copy(b.begin(),b.end(),back_inserter(end->m_goto));
}
\end{verbatim}
For {\tt{var01}} in figure \ref{expr_e013},
the way of generating {\tt{do-while}} statement 3 address code
becomes like bellow.
\begin{verbatim}
// Override virtual function
void var01::do_code(do_stmt* info, to3ac* begin)
{
  // Replace the codes like bellow
  //
  //   if y op z goto label
  //   res := 1             # This position is recorded in `m_one'
  //   goto end
  // label:
  //   res := 0             # This position is recorded in `m_zero'
  // end:
  //
  // to like bellow
  //
  // begin:                   # Already generated.
  //   (Code of info->m_stmt) # Already generated.
  //   if y op z goto label
  //   goto begin
  // label:
}
\end{verbatim}
We mentioned about the way of generating codes for
{\tt{do-while}} statement at \ref{expr_e022}, where
we said that it was better to distingish 
the way for {\tt{var01}} and that for {\tt{log01}} in
figure \ref{expr_e013}.

For {\tt{log01}},
the way of generating {\tt{do-while}} statement 3 address code
becomes like bellow.
\begin{verbatim}
// Override virtual function
void log01::do_code(do_stmt* info, to3ac* begin)
{
  // Replace the codes like bellow
  //
  //   if y1 op z1 goto label
  //   if y2 op z2 goto label
  //   ret := 1
  //   goto end
  // label:
  //   ret := 0
  // end:
  //
  // to like bellow
  //
  // begin:                   # Already generated
  //   (Code of info->m_stmt) # Already generated
  //   if y1 op z1 goto label
  //   if y2 op z2 goto label
  //   goto begin
  // label:
}
\end{verbatim}
For arithmetic constant and constant pointer,
the way of generating {\tt{do-while}} statement 3 address code
becomes like bellow.
\begin{verbatim}
template<class T> void
common(constant<T>* ptr, do_stmt* info, to3ac* begin);

// Override virtual function
void constant<char>::do_code(do_stmt* info, to3ac* begin)
{ common(this,info,begin); }
...
void constant<long double>::do_code(do_stmt* info, to3ac* begin)
{ common(this,info,begin); }
void constant<void*>::do_code(do_stmt* info, to3ac* begin)
{ common(this,info,begin); }

template<class T> void
common(constant<T>* ptr, do_stmt* info, to3ac* begin)
{
  if ( !ptr->zero() ) {
    // Generate jump to the loop head
    goto3ac* go = new goto3ac;
    go->m_to = begin;
    begin->m_goto.push_back(go);
    code.push_back(go);
  }

  // Update destination address according to break statement
  ...
}
\end{verbatim}

\subsection{{\tt{for}} statement}

Considering the case replacing first expression to 
some declaration,
the syntax analisis tree can be represented like bellow.
\begin{verbatim}
struct for_stmt : stmt, break_outer, continue_outer {
  scope* m_scope;
  vector<tac*> m_decl;  // Initializer code for some declaration
  expr* m_expr1;
  expr* m_expr2;
  expr* m_expr3;
  base* m_stmt;
  ...
  void gen()
  {
    struct x {
      scope* m_org;
       x(){ m_org = scope::current; }
      ~x(){ scope::current = m_org; }
    } x;
    scope::current = m_scope;
    if ( m_expr1 )
      m_stmt1->gen();
    else
      copy(m_decl.begin(),m_decl.end(),back_inserter(code));
    to3ac* begin = new to3ac;
    code.push_back(begin);
    var unused(0);
    var* expr = m_expr2 ? m_expr2->gen() : &unused;
    expr->for_code(this,begin);  // Call virtual function
                                 // for generating the rest code.
  }
};
\end{verbatim}
{\tt{var::for\_code}} which generates most general codes
becomes like bellow.
\begin{verbatim}
void var::for_code(for_stmt* info, to3ac* begin)
{
  goto3ac* goto2 = 0;
  if ( m_type ) {
    // The second expression is specified.
    var* expr2 = rvalue();
    expr2 = expr2->promotion();
    const type* T = expr2->m_type;
    if ( !T->scalar() ) {
      // Not scalar. Error.
    }
    var* zero = integer::create(0);
    zero = zero->cast(T);

    // Generate code like if expression equals to zero then leave loop.
    goto2 = new goto3ac(goto3ac::EQ,expr2,zero);
    code.push_back(goto2);
  }

  break_stack.push(info);
  continue_stack.push(info);
  info->m_stmt->gen();
  continue_stack.pop();
  break_stack.pop();

  to3ac* to3 = new to3ac;
  code.push_back(to3);

  // Update destination address according to continue statement
  continue_stmt::outer& c = *info;
  for_each(c.begin(),c.end(),bind2nd(ptr_fun(update),to3));
  copy(c.begin(),c.end(),back_inserter(to3->m_goto));

  if ( info->m_expr3 ) {
    var* expr3 = info->m_expr3->gen();
    expr3->rvalue();
  }

  // Generate code like jump to the loop head.
  goto3ac* goto1 = new goto3ac;
  code.push_back(goto1);
  begin->m_goto.push_back(goto1);
  goto1->m_to = begin;

  to3ac* to2 = new to3ac;
  code.push_back(to2);

  // Update destination address according to break statement
  break_stmt::outer& b = *info;
  for_each(b.begin(),b.end(),bind2nd(ptr_fun(update),to2));
  copy(b.begin(),b.end(),back_inserter(to2->m_goto));

  if ( goto2 ) {
    to2->m_goto.push_back(goto2);
    goto2->m_to = to2;
  }
}
\end{verbatim}
For {\tt{var01}} in figure \ref{expr_e013},
the way of generating {\tt{for}} statement 3 address code
becomes like bellow.
\begin{verbatim}
// Override virtual function
void var01::for_code(for_stmt* info, to3ac* begin)
{
  // Replace the codes like bellow
  //
  //   if y op z goto label
  //   res := 1             # This position is recorded in `m_one'
  //   goto end
  // label:
  //   res := 0             # This position is recorded in `m_zero'
  // end:
  //
  // to like bellow
  //
  // begin:                     # Already generated.
  //   (Code of info->m_expr1)  # Already generated.
  //   or                       # Already generated.
  //   (info->m_decl)           # Already generated.
  //
  //   if y op z goto label
  //   (Code of info->m_stmt)
  // cont:                      # Continue destination address
  //   (Code of info->m_expr3)
  // label:                     # Break destination address
}
\end{verbatim}
For arithmetic constant and constant pointer,
the way of generating {\tt{for}} statement 3 address code
becomes like bellow.
\begin{verbatim}
template<class T> void
common(constant<T>* ptr, for_stmt* info, to3ac* begin);

// Override virtual function
void constant<char>::for_code(for_stmt* info, to3ac* begin)
{ common(this,info,begin); }
...
void constant<long double>::for_code(for_stmt* info, to3ac* begin)
{ common(this,info,begin); }
void constant<void*>::for_code(for_stmt* info, to3ac* begin)
{ common(this,info,begin); }

template<class T> void
common(constant<T>* ptr, for_stmt* info, to3ac* begin)
{
  if ( !ptr->zero() ) {
    // Generate code as if second expression is not specified
    var unused(0);
    unused.for_code(info,begin);
  }
  else {
    int n = code.size();
    info->m_stmt->gen();  // Once, gnerate 3 address code
    code.resize(n);       // and discard them
  }
}
\end{verbatim}

\section{\tt{jump-statement}}

\subsection{goto statement}
\label{stmt_e001}
The syntax analisis tree can be represented like bellow.
\begin{verbatim}
struct goto_stmt : stmt {
  string m_label;
  ...
  void gen();
};
\end{verbatim}
At \ref{stmt_e000}, we mentioned about defined and used label tables.
Here, frontend references these tables.
\begin{verbatim}
void goto_stmt::gen()
{
  map<string,define>::const_iterator p = defined.find(m_label);
  if ( p == defined.end() ) {
    // Not defined label referenced.
    goto3ac* go = new goto3ac;
    code.push_back(go);
    used[m_label].push_back(use(go,...));  // Insert into table
    return;
  }

  // Already defined label referenced.
  const label::define& d = p->second;
  to3ac* to = d.m_to;
  goto3ac* go = new goto3ac;
  code.push_back(go);
  to->m_goto.push_back(go); // Mark the label referencedy
                            // by this `goto3ac'
  go->m_to = to;
}
\end{verbatim}

\subsection{continue statement}

The syntax analisis tree can be represented like bellow.
\begin{verbatim}
struct continue_stmt : stmt {
  ...
  void gen()
  {
    if ( continue_stack.empty() ) {
      // Outside loop. Error.
    }
    outer* top = continue_stack.top();
    goto3ac* go = new goto3ac;
    top->push_back(go);
    code.push_back(go);
  }
};
\end{verbatim}
Here, {\tt{continue\_stack}} is the same
with that of \ref{stmt_e004}. 

\subsection{break statement}

The syntax analisis tree can be represented like bellow.
\begin{verbatim}
struct break_stmt : stmt {
  ...
  void gen()
  {
    if ( break_stack.empty() ) {
      // Outside switch statement or loop. Error.
    }
    outer* top = break_stack.top();
    goto3ac* go = new goto3ac;
    top->push_back(go);
    code.push_back(go);
  }
};
\end{verbatim}
Here, {\tt{break\_stack}} is the same
with that of \ref{stmt_e005}.

\subsection{return statement}

The syntax analisis tree can be represented like bellow.
\begin{verbatim}
struct return_stmt : stmt {
  expr* m_expr;
  ...
  void gen();
};
\end{verbatim}
The return statment is valid if and only if the assignment is valid for the
expression.
{\tt{return\_stmt::gen}} becomes like bellow.
\begin{verbatim}
void return_stmt::gen()
{
  var* expr = 0;
  if ( m_expr ) {
    expr = m_expr->eval();
    expr = expr->rvalue();
  }

  // Get current function return type.
  const type* T = fundef::current->m_usr->m_type;
  typedef const func_type FUNC;
  FUNC* func = static_cast<FUNC*>(T);
  T = func->return_type();

  if ( expr ) {
    T = assign_impl::valid(T,expr);
    if ( !T ) {
      // Assignment is not valid. i.e. This return statemnet
      // is not valid. Error.
    }
    expr = expr->cast(T);
  }
  else {
    const type* vt = void_type::create();
    if ( !T->compatible(vt) ) {
      // The current function return type is not `void'.
      // Error.
    }
  }
  code.push_back(new return3ac(expr));
}
\end{verbatim}
