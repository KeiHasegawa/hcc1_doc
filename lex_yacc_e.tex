\label{lex_yacc_e000}
\index{lexical analizer@lexical analizer}
\index{syntactic analysis@syntactic analysis}
As described in chpater \ref{front_back_e000}, frontend
converts program to 3 address code and symbol table.
Before this converting, frontend does lexixcal analisis
and syntax analisis. Here we described about these.

Bibliography \cite{doragon} illustrates precicely
the way of lexical analisis and syntax analisis.
Now we'll use {\tt{lex}} and {\tt{yacc}} because
they are very useful.

We can find the C language grammer in bibliography \cite{ISO}.
In bibliography \cite{KR}, \cite{SGS}, you can find 
the same kind of description.
And it is very simple to convert {\tt{yacc}} program text.
We also can find the C laugage lexicograph in bibliography \cite{ISO}.
Unfortunately, it's not simple to convert {\tt{lex}} program text, so,
it's necessary to convert it to regular expression.
%%\begin{htmlonly}
%%There are  {\tt{lex}} program text and {\tt{yacc}} program text.
%%in \ref{grammer_e000}.
%%\end{htmlonly}

\section{Syntax analisis using {\tt{yacc}} }

\label{lex_yacc_e003}
You can find that {\tt{yacc}} just report shift/reduce conflict about
{\tt{if else}}. And this conflict is recognized as `shift', and it doesn't
matter.

In the process of upper syntax analisis, we'll just make
syntax tree. For example, about {\tt{additive-expression}}

\begin{verbatim}
/* yacc program text */
%type<expr> multiplicative_expression additive_expression
...
additive_expression
  : multiplicative_expression
  | additive_expression '+' multiplicative_expression
    { $$ = new bin_expr('+',$1,$3); }
  | additive_expression '-' multiplicative_expression
    { $$ = new bin_expr('-',$1,$3); }
  ;

// Node for expression
class expr {
  ...
  // Interface for generating 3 address code
  virtual var* eval() = 0;
};

// Node for binary expression
class bin_expr : public expr {
  ...
  bin_expr(int op, expr* left, expr* right);
  var* eval();
};
\end{verbatim}

\begin{QandA}
Generally, for grammer ``A $\rightarrow$ B $|$ C D'', define
structure Sa, Sb, Sc like below
\begin{verbatim}
struct Sb;
struct Sc;
struct Sd;
struct Sa {
  Sb* m_Sb;
  Sc* m_Sc;
  Sd* m_Sd;
  Sa(Sb* b) : m_Sb(b), m_Sc(0), m_Sd(0) {}
  Sa(Sc* c, Sd* d) : m_Sb(0), m_Sc(c), m_Sd(d) {}
};

A : B   { $$ = new Sa($1); }
  | C D { $$ = new Sa($1,$2); }
  ;
\end{verbatim}
This way for generating syntax tree is simple and automatic.
Does it work?

Answer : It works but please take care if grammer is defined
recursively like programming lauguage C expression. For example,
a + b syntax tree contains many empty node. That is, in above
grammer, almost all are ``A $\rightarrow$ B''.
It cause use of large heap and then compiler cannot run fast.
\end{QandA}

\begin{QandA}
\label{lex_yacc_e001}
Is it OK not to generate syntax tree but to generate 3 address
code immediately in the process of upper syntax analisis.

Answer : It's OK. But in this case, you must move a 
group of 3 address codes. For example, {\tt{for}} statement
code generation is implemented like below if you generate
syntax tree in the process of upper syntax analisis.

\begin{verbatim}
/* yacc program */
for_statement
  : ...
  | FOR '(' expression ';' expression ';' expression ')' statement
    { $$ = new for_stmt($3,$5,$7,$9); }
  ;

// for statement code generation
void for_stmt::gen()
{
  generate 3 address code of $3
  label("begin");
  generate 3 address code of $5
  generate if result of $5 == 0 goto "end"
  generate 3 address code of $9
  generate 3 address code of $7
  generate goto "begin"
  label("end);
}
\end{verbatim}

On the other hand, if you generate 3 address code immediately in the
process of upper syntax analisis,
you must replace 3 address code of {\tt{\$7}}
and that of {\tt{\$9}}.
\end{QandA}

\section{Lexical analisis using {\tt{lex}}}
\label{lex_yacc_e004}
As described in \ref{lex_yacc_e003}, if we can use  {\tt{yacc}},
syntax analisis of C programming lauguage is not so difficult.
But please remember that lexical analizer must work well for it.
The lexical analisis point is to chose a correct token
for a lexicography which matches
regular expression
\begin{verbatim}
{nondigit}({nondigit}|{digit})*
\end{verbatim}
For this, compiler must reference symbol table at least. For example,

\begin{verbatim}
typedef int A;    /* insert `A' into symbol table. */
A a;              /* `A' isn't identifier. It's 
                     typedef-name. */

void f(void)
{
  A A = A;        /* 1st `A' is typedef-name. */
                  /* 2nd `A' isn't typedef-name. It's
                     identifier. In this case, compiler
                     doesn't look up symbol table. */
                  /* For 3rd `A', compiler looks up 
                     symbol table, and it's identifier. */
  ...
}
\end{verbatim}
When compiler detects the minimum unit of declaration,
compiler must insert it into symbol table.

\begin{verbatim}
declaration
  : declaration_specifiers init_declarator_list ';'
    { /* incorrect timing. */ }
  | ...
  ;
...
init_declarator
  : declarator     { /* Now insert symbol table. */ }
  | declarator '=' { /* Now insert symbol table. */ } initializer
  ;
...
\end{verbatim}
When compiler choses a correct token, compiler looks up symbol
table, so, the attribute of lexicography {\tt{identifier}}, {\tt{typedef-name}}
should be an entry of symbol table.
Note that compiler doesn't have to look up in {\tt{yacc}} action part.
Please reference bibliography \cite{doragon} about the attribute
of lexicography.

\begin{QandA}
Bibliography \cite{doragon} says that
the attribute of lexicography {\tt{constant}}
should be an correspond constant. But in figure
\ref{front_back_e007}, constant integer {\tt{0}}
and string literal {\tt{``hello world\verb|\|n''}}
are in symbol table. Is it necessary? What attribute 
should be for {\tt{constant}}?

Answer : {\tt{constant}} has it's type in C launguage.
And string literal storage class is {\tt{static}}.
Please reference bibliography \cite{ISO} about these.
Every constant has it's name and type, so it is natural
that compiler inserts constant into symbol table.
The attribute of lexicography {\tt{constant}} 
should be an entry of symbol table like {\tt{identifier}}
or {\tt{typedef-name}}.
\end{QandA}

\subsection{Lexical analyzer and symbol table}
\label{lex_yacc007_e}
In \ref{lex_yacc_e004}, we described that comiler must
look up symbol table for a lexicography
which matches to regular expression
\begin{verbatim}
{nondigit}({nondigit}|{digit})*
\end{verbatim}
in some cases. If you use the grammer of bibliography \cite{ISO},
{\tt identifier} and {\tt typedef-name} matches to this regular
expression. But not only variable and typedef name are in symbol table.
Tag name of structure, union and enumeration are also in symbol table.
So we'll let {\tt tag-name} be a terminal symbol and change the
grammer of bibliography \cite{ISO} like below.

\begin{verbatim}
struct-or-union-specifier :
  struct-or-union identifier { struct_declaration_list }
  struct-or-union { struct-declaration-list }
  struct-or-union identifier
  struct-or-union tag-name              # This rule is added.
...
enum-specifier:
  enum identifier { enumerator-list }
  enum            { enumerator-list }
  enum identifier { enumerator-list , }
  enum            { enumerator-list , }
  enum identifier
  enum tag-name                         # This rule is added.
\end{verbatim}
In the rules of {\tt struct-or-union-specifier },
1st and 2nd rule are the declaration of complete structure or union,
3rd rurle is the declaration of incomplete structure or union,
4th rule is the reference of already defined structure or union.
In the below example, we'll know that compiler must look up
symbol table only in necessary case.
\begin{verbatim}
typedef int T;  /* Not look up. `T' is identifier. */

T t;  /* Look up and decide `T' is typedef-name. */

struct T {  /* Not look up. `T' is identifier. */
  ...
};

void f(void)
{
  struct T t; /* Look up and decide `T' is tag-name. */
  struct T {  /* Not look up. `T' is identifier. */ 
    ...
  };
}
\end{verbatim}
Here, we'll think the case compiler should look up symbol table
or not when a lexicography matches to regular expression
{\tt \{nondigit\}(\{nondigit\}|\{digit\})*}.

\newtheorem{Method}{Method}[chapter]

\begin{Method}
\label{lex_yacc_e002}
Judge from previous token.

\begin{verbatim}
int a;          /* previous is int, so not look up.
                   `a' is identifier. */
typedef int A;  /* previous is int, so not look up.
                   `A' is identifier. */
A aa;           /* previous is ';', so look up and
                   `A' is typedef-name. */
                /* previous is typedef-name, so not look up.
                   `aa' is identifier. */
\end{verbatim}

If previous token is one of
\begin{verbatim}
void char int float double short long signed unsigned
typdef-name tag-name . -> goto
\end{verbatim}
then not look up and make it  {\tt{identifier}}.
\end{Method}

There are some problems in method \ref{lex_yacc_e002} because
compiler cannot decide to look up or not from
previous token.

\begin{verbatim}
int const a;  /* previous is const,
                 not look up?  */
typedef int A;
const A b;    /* Not good in this case. For `A',
                 compiler must look up. */
\end{verbatim}
If previous token is one of
\begin{verbatim}
volatile restrict typedef extern static register auto inline
\end{verbatim}
the same problem with {\tt{const}} occurs.

Other problem will occur when the token
should be {\tt{tag-name}}.

\begin{verbatim}
struct S {  /* previous is struct, not look up? */
  ...
};

struct S s; /* Not good. For `S', compiler must look up. */
\end{verbatim}


\begin{Method}
\label{lex_yacc_e005}
Samely as Method \ref{lex_yacc_e002}, remember previous token.
And also remember the return values of lexical analyzer
which is one of
\begin{verbatim}
void char int float double short long signed unsigned
typdef-name tag-name
const volatile restrict
typedef extern static register auto inline
\end{verbatim}
For these token, prepare global container like below.
\begin{verbatim}
vector<int> lexed_spec;
\end{verbatim}
This container must be cleared when one of
\begin{verbatim}
declaration
struct-declaration
parameter-declaration
\end{verbatim}
is reduced.

Below function tells us whether compiler should lookup
symbol table or not.
\begin{verbatim}
bool should_lookup()
{
  if ( prev == STRUCT || prev == UNION || prev == ENUM )
    return peek() != '{';

  if ( lexed_spec.empty() )
    return prev != '.' && prev != ARROW && prev != GOTO;

  return find_if(lexed_spec.begin(),lexed_spec.end(),enough)
    == lexed_spec.end();
}

bool enough(int n)
{
  If n is one of below, return true.
  void char int float double short long signed unsigned
  typdef-name tag-name

  If not, return false. i.e. n is one of 
  const volatile restrict
  typedef extern static register auto inline
}
\end{verbatim}
\end{Method}

There are some problems in method \ref{lex_yacc_e005}.

\begin{verbatim}
int typedef A;      /* A : Not look up because int and typedef
                       are in lexed_spec. */

A a[sizeof(const A)], b;
                    /* 1st `A' : Look up because lexed_spec is empty
                       and previous token is ';'. */
                    /* `a' : Not look up because typedef-name is
                       in lexed_spec. */
                    /* 2nd `A' : Not good. Compiler should have
                       looked up. typedef-name and const are
                       in lexed_spec. */
                    /* `b' : Not look up because typedef-name
                        and const are in lexed_spec. This is OK,
                        but type of `b' becomes const A. */

void f(A a);        /* `f' : Not look up because void is in lexed_spec. */
                    /* `A' : Not good. Compiler should have looked up.
                        void is in lexed_spec. */
                    /* `a' : Not look up because void is in lexed_spec.
                       This is OK, but type of `a' becomes void. */
\end{verbatim}

\begin{Method}
\label{lex_yacc_e006}
Samely as \ref{lex_yacc_e002}, remember previous token.
And samely as \ref{lex_yacc_e005}, remember {\tt{lexed\_spec}}.
And more, remember below infomation.
\begin{verbatim}
struct spec {
  type* m_type; // remember type
  int m_other;  // remember storage class or function specifier
};
stack<spec*> spec_stack;
\end{verbatim}
Before clearing {\tt{lexed\_spec}}, 
push {\tt{spec}} whihch is same with {\tt{lexed\_spec}}
to {\tt{spec\_stack}}.
And then push 0.
Below example shows the timing of clear, push and pop.

\begin{verbatim}
typedef short int A[3];
static int n =
/* push (static, int). clear lexed_spec. push 0. */
sizeof(A), /* `A' : Look up. pop. */
m    /* Reference spec_stack.top() */
;    /* pop. */

struct S {
  unsigned int a :
  /* push (unsigned int). clear lexed_spec. push 0. */
  sizeof(A), /* `A' : Look up. pop. */
  b[5]  /* Reference spec_stack.top() */
  ;     /* pop. */
};

int b[  /* push (int). clear lexed_spec. push 0. */
     sizeof(A)  /* `A' : Look up. */
     ]  /* pop */
     , c /* Reference spec_stack.top(). */
     ;     /* pop. */

inline void f(a,b,c)
  /* push (inline, void ()). clear lexed_spec. push 0. */
  int a;    /* `a' : Not look up because int is in lexed_spec. */
  double b; /* `b' : Not look up because double is in lexed_spec. */
  char* c;  /* `c' : Not look up because spec_stack.top() has infomation. */
  /* pop */
{
  ...
}

static void g(
/* push (static, void). clear lexed_spec. push 0. */
  FILE  /* Look up because spec_stack.top() is 0. */
  *
  fp    /* Not look up because spec_stack.top() is FILE*.  */
  /* pop */
);
\end{verbatim}

Now, we can rewrite function {\tt{should\_lookup}} like below.
\begin{verbatim}
bool should_lookup()
{
  if ( prev == STRUCT || prev == UNION || prev == ENUM )
    return peek() != '{';

  if ( !lexed_spec.empty() ) {
    return find_if(lexed_spec.begin(),lexed_spec.end(),enough)
      == lexed_spec.end();
  }

  if ( !spec_stack.empty() )
    return !spec_stack.top();

  return prev != '.' && prev != ARROW && prev != GOTO;
}
\end{verbatim}
\end{Method}

There is a problem about {\tt{abstract-declarator}}
in method \ref{lex_yacc_e006}.
\begin{verbatim}
typedef int T;
T f(T (T));
\end{verbatim}
where {\tt{f}} should not be same with
\begin{verbatim}
T f(T);
\end{verbatim}
But {\tt{f}} should be same with
\begin{verbatim}
T f(T (*)(T));
\end{verbatim}
In 1st declaration of {\tt{f}}, 3rd {\tt{T}} should have been
looked up and decided it {\tt{typedef-name}}.
Please note that the declaration
\begin{verbatim}
T f(T (a));
\end{verbatim}
is equivalent to
\begin{verbatim}
T f(T);
\end{verbatim}
And also, please note that
\begin{verbatim}
T (T);  // T (T) out side of parameter
\end{verbatim}
is equivalent to
\begin{verbatim}
int T;
\end{verbatim}
and
\begin{verbatim}
T f(T (T));  // T (T) at parameter scope
\end{verbatim}
is equivalent to
\begin{verbatim}
int f(int (*pf)(int));
\end{verbatim}

Above example shows that
we cannot decide whether compiler should look up or not.
But we decide like below.
If previous token is '(' and compiler analyzes
in parameter scope, then look up the symbol. And
if it is {\tt{typedef-name}}, its token is {\tt{typedef-name}}.
That is,

\begin{verbatim}
/* lex program */
{nondigit}({nondigit}|{digit})*   { return judge(yytext); }

// Decide token of name. identifier, typedef-name or tag-name.
int judge(string name)
{
  if ( prev == STRUCT || prev == UNION || prev == ENUM ) {
    if ( peek() == '{'; )
      return IDENTIFIER;
    if ( int r = lookup(name) )  // Only look up tag name
      return r;
    return IDENTIFIER;
  }

  if ( !lexed_spec.empty() ) {
    p = find_if(lexed_spec.begin(),lexed_spec.end(),enough);
    if ( p != lexed_spec.end() )
      return IDENTIFIER;
    if ( int r = lookup(name) )
      return r;
    return IDENTIFIER; // If not exists, not error.
  }

  if ( !spec_stack.empty() ) {
    if ( prev == '(' && analyzing parameter scope ) {
      int r = lookup(name);
      if ( r && r == TYPEDEF_NAME )
        return TYPEDEF_NAME;
      return IDENTIFIER; // If not exists, not error.
    }

    if ( spec_stack.top() )
      return IDENTIFIER;

    if ( int r = lookup(name) )
      return r;   

    return IDENTIFIER; // Maybe error, but cannot decide here.
  }

  if ( prev == '.' || prev == ARROW || prev == GOTO )
    return IDENTIFIER;

  if ( int r = lookup(name) )
    return r;

  return IDENTIFIER; // Maybe error, but cannot decide here.
}
\end{verbatim}

Compiler cannot make it error immediately even if the symbol is not in
symbol table.
\begin{verbatim}
main()  /* Not specified return type or storage class of `main'.
           lexed_spec is empty and spec_stack is empty.
           previous token is meaningless. Compiler looks up
           but not in symbol table. */
{
  printf("hello world\n");
  /* call function which is not declared.
     lexed_spec is empty and spec_stack is
     empty. previous token is '{'. Compiler looks up
     but not in symbol table. */

label:   /* Look up but not in symbol table. */
  ;

  int f(a,b,c);  /* For `a', `b', `c',
                    look up but not in symbol table. */
}
\end{verbatim}
In some cases, undeclared variable reference in expression 
must be dealt when compiler generates 3 address code.

\begin{verbatim}
int f();

int g(int a)
{
  // ...
label:      // Should be looked up `label'?
  f();

  // ...
  a ? --a : g(++a);   // ':' apears here

  struct S {
    int a : sizeof f();  // And ':' apears here too.

    // ...
  };

f:   // should be look up `f'?
  g(f());

  // ...
}
\end{verbatim}
Finaly, please consider that it sould be looked up identifier
before ':'.
The rules we discussed at this section specify that:
\begin{itemize}
\item
look up `a' at expr1 ? a : expr2 
\item
not look up `a' at int a : sizeof f();
\end{itemize}

I don't want to, if it's possible, look up label at
labeled-statement. But if it's difficult to judge,
just looking up may be simple because
we can change the way to deal with the label
at code generation routine of labeled-statement,
according to the label entried at symbol table or not.
