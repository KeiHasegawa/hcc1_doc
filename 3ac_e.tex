\label{_3ac_e001}
Chapter 8 of bibliography \cite{doragon} says that
which kinds of 3 address code are necessary.
Here, we'll think about the set of 3 address code for C compiler.
3 address code can be coded like below structure.
\index{3 address code@3 address code}
\begin{verbatim}
struct var;  // variable. entry of symbol table.

struct tac {
  var* x;
  var* y;
  var* z;
  tac(var* xx, var* yy, var* z) : x(xx), y(yy), z(zz) {}
};
\end{verbatim}
For examle, 3 address code of addition is
derived from {\tt{tac}} like below.
\begin{verbatim}
struct add3ac : tac {
  add3ac(var* x, var* y, var* z) : tac(x,y,z) {}
};
\end{verbatim}
Chapter 8 of bibliography \cite{doragon} says that
it's necessary to distinguish floating-point addition and
integer addition. In our case, it's possible to distinguish
them from {\tt {m\_type}} member of operand. i.e.
\begin{verbatim}
void f(add3ac* p)
{
  const type* T = p->x->m_type;
  if ( T->real() )
    ...  // floating-point addtion
  else
    ...  // integer addtion
}
\end{verbatim}
So, we'll generate {\tt{add3ac}} for both floating-point addition
and integer addition. Frontend must generate 3 address code
of which operands make sense.

When {\tt{function-definition}} is reduced, frontend generates
3 address codes. For example, for a piece of program code
\begin{verbatim}
extern int printf(const char*, ...);

int f(void)
{
   int a = 1;
   a = (a + 2) * 3;
   return printf("a = %d\n",a);
}
\end{verbatim}
frontend generates 3 address codes like below.
\begin{verbatim}
f:
        a := 1
        t0 := a + 2
        t1 := t0 * 3
        a := t1
        t2 := &"a = %d\n"
        param t2
        param a
        t3 := call printf
        return t3
\end{verbatim}

\begin{QandA}
How do we use members of {\tt{tac}}?

Answer : In optimization phase of 3 address codes, we want to
investigate usage or definition of operands. See chapter 10 in
bibliography \cite{doragon} about ud-chain and du-chain.
{\tt{x}} denotes the definition and {\tt{y, z}} denote the usage.
\end{QandA}

\begin{QandA}
I want add virtual function into {\tt{tac}} for generating target
code. Is it OK?

Answer : Sorry, but it isn't OK because frontend doesn't have to
generate target code and frontend and backend must be build in
implementation. I think you want to do like below.
\begin{verbatim}
struct tac {
  ...
  virtual void gen_target() const = 0;
};

void backend::tac2target(const vector<tac*>& code)
{
  // convert a 3 address code to some target codes automatically.
  for_each(code.begin(),code.end(),mem_fun(&tac::gen_target));
}
\end{verbatim}
Instead of above, you can code like below.
\begin{verbatim}
struct backend::table : map<string, void (*)(const tac*)> {
  table()
  {
    (*this)[typeid(assign3ac).name()] = assign;
    (*this)[typeid(add3ac).name()] = add;
    (*this)[typeid(sub3ac).name()] = sub;
    (*this)[typeid(mul3ac).name()] = mul;
    ...
  }
};

void backend::tac2target(const vector<tac*>& code)
{
  for_each(code.begin(),code.end(),gencode);
}

void backend::gencode(const tac* p)
{
  table[typeid(*p).name()](p);
}

void backend::assign(const tac* p)
{
  // Do the same work with assign::gen_target, please.
}
\end{verbatim}
\end{QandA}

\section{{\tt{x := y}}}

A class coresspond to {\tt{x := y}} can be defined like below.
\begin{verbatim}
struct assign3ac : tac {
  assign3ac(var* x, var* y) : tac(x,y,0) {}
};
\end{verbatim}
The type of operand is one of scalar, structure or union. But not
array because assignment of array is not defined in C language.
Backend can distingish various assignments from the type of {\tt{x}}.
Frontend must generates {\tt{x := y}} of which operands make sense.

\section{{\tt{x := y * z}}}

Multiplications in C language are defined for arithmetic type operands.
So the type of {\tt{y}} or {\tt{z}} must be arithmetic.
Backend can distingish various multiplications from the type of {\tt{x}}.
Frontend must generates {\tt{x := y * z}} of which operands make sense.

\section{{\tt{x := y / z}}}

For divisions, samely as multiplications,
backend can distingish various divisions from the type of {\tt{x}}.
Frontend must generates {\tt{x := y / z}} of which operands make sense.

\begin{QandA}
For {\tt{x := y}}, is the type of {\tt{x}} {\it compatible}
\index{compatible@compatible}
with that of {\tt{y}}?  Samely, for 
{\tt{x := y * z}} and {\tt{x := y / z}},
the types of {\tt{x}}, {\tt{y}} and {\tt{z}} are
{\it compatible} with each other?

Answer : It's not, because the none-{\it compatible} condition
is effective considering optimization. For example,
\begin{verbatim}
long int f(unsigned int a, int b){ return a * b; }
\end{verbatim}
For above function, fronend may generate 3 address codes
like below.
\begin{verbatim}
f:
    t0 := (unsigned int)b    # t0 : unsigned int
    t1 := a * t0             # t1 : unsigned int
    t2 := (long int)t1       # t2 : long int
    return t2
\end{verbatim}
According to the specification of C language,
arithmetic conversion and {\tt{return}} statement conversion
will be ocurred, and above 3 address codes are correct.
But frontend can know that converion from {\tt{int}} to {\tt{unsigned int}}
is equivalent to assignment, and the existence of sign
does not affect the resut of integer multiplication.
And frontend can eliminate first conversion code.
And more, if backend assigns the same space
to {\tt {int}} and {\tt {long int}}, and if frontend can know
that, third converions code can be eliminated.
In this case, optimized 3 address codes are like below.
\begin{verbatim}
f:
    t1 := a * b             # t1 : unsigned int
    return t1
\end{verbatim}
But please take care if {\tt{f}} returns the result of division
instead of multiplication. In this case, frontend cannot convert
samely. Generally, not only operand size, but
also the existence of sign affects the result of integer
division. If backend assigns the same space
to {\tt {int}} and {\tt {long int}}, and if frontend can know
that, frontend can eliminate arithmetic conversion
for division {\tt {int}} and {\tt {long int}}.
\end{QandA}

\section{{\tt{x := y \% z}}}

Moduli are defined for integer operand.
Backend can distingish various moduli from the type of {\tt{x}}.
Frontend must generates {\tt{x := y \% z}} of which operands make sense.

\section{{\tt{x := y + z}}}

Additions are defined for arithmetic type operands.
And also defined for pointer to object type
and integer, and integer and pointer to object type.
In this case, the result of addition is the pointer type.
Frontend must generates {\tt{x := y + z}} of which operands make sense.

\section{{\tt{x := y - z}}}

Subtraction are defined for arithmetic type operands.
And also defined for pointer to object type
and integer. But not defined for integer and pointer to object type.
And more, defined for pointer to object type and pointer that is
compatible with it. The result type is {\tt{long int}}.

\section{{\tt{x := y << z}}, {\tt{x := y >> z}}}

Left shift first operand is integer type. Second operand is {\tt{int}}.
Normal arithmetic conversion is not applied to left shift.
The result type is the same type of first operand.

Right shift operands are restricted samely with left shift.
Normally, if first operand type has sign, right shift 
is implemented as arithmetic shift. otherwise right shift
is equal to logical shift. But it depends on implementation.

\section{{\tt{x := y \& z}}, {\tt{x := y \^{} z}}, {\tt{x := y | z}}}

Bitwise and, exclusive-or, or are defined for integer operands.

\section{{\tt{x := -y}}, {\tt{x := \~{}y}}}

Unary minus is defined for arithmetic operand.
Bitwise-not is defined for integer operand.

\section{{\tt{x := (type)y}}}

The class according to 3 address code for type conversion is like
below.
\begin{verbatim}
struct cast3ac : tac {
  const type* m_type;   // contain type
  cast3ac(var* x, var* y, const type* t) : tac(x,y,0), m_type(t) {}
};  
\end{verbatim}

\section{{\tt{x := \&y}}}

The type of {\tt{x}} is pointer to the type of {\tt{y}}.

\begin{QandA}
Unary {\tt{\&}} operator is defined for {\it lvalue}.
Does {\tt {y}} have {\it lvalue} in {\tt{x := \&y}}?

Answer : No. First, note that 3 address code is not
equal to programming language C. For other programming
languages, the same 3 addres code set described here
can be used for implementing compiler. Frontend generates
3 address code of which operands make sense, in every case.
For medium variable {\tt{y}} which doesn't have {\it lvaue},
compiler may generate {\tt{x := \&y}}.
\end{QandA}

\section{{\tt{x := *y}}}

The type of {\tt{y}} must be pointer to object type,
and the type of {\tt{x}} is the object type.
Backend can distingish various {\tt{x := *y}} from the type of {\tt{x}}.

\section{{\tt{*y := z}}}

For both operand {\tt{y}} and {\tt{z}}, we regard source operands.
The structure according this 3 address code is like below.
\begin{verbatim}
struct invladdr3ac : public tac {
  invladdr3ac(var* y, var* z) : tac(0,y,z) {}
};
\end{verbatim}
The type of {\tt{y}} must be pointer to object type,
and the type of {\tt{z}} is the object type.
Backend can distingish various {\tt{*y := z}} from the type of {\tt{z}}.

\section{\tt{x[y] := z}, \tt{x := y[z]}}
\label{3ac_e004}
For {\tt{x[y] := z}}, the type of {\tt{x}} is aggregate, i.e.
one of structure, union and array.
Samely, for {\tt{x := y[z]}}, the type of {\tt{y}} is aggregate.
These 3 addres codes may be
generated for member reference or subscripting operator for array.
Backend can distingish various {\tt{x[y] := z}} from the type of
{\tt{z}}, and {\tt{x := y[z]}} from the type of {\tt{x}}.

\section{{\tt{x := call y}}, {\tt{call y}}, {\tt{param y}}}

\label{_3ac_e000} These are 3 address code for function call.
The structure according to these are like below.
\begin{verbatim}
struct param3ac : tac {
  param3ac(var* y) : tac(0,y,0) {}
};

struct call3ac : tac {
  call3ac(var* x, var* y) : tac(x,y,0) {}
};
\end{verbatim}
Frontend must generate {\tt{param}} or {\tt{param}}s continually
which is or are referenced by {\tt{call}}, just before generating
{\tt{call}}.

For {\tt{x := call y}}, the type of {\tt{x}} is one
of scalar, structure or union.
The type of {\tt{y}} is function or pointer to function.
If member {\tt{x}} is zero-pointer, it denotes that
function return value is not used.

For {\tt {param y}}, the type of {\tt{y}} is one of
scalar, structure or union.

Backend can distingish various 
{\tt{x := call y}}, {\tt{call y}} or {\tt{param y}}
from the type of operands.

\section{{\tt{return}}, {\tt{return y}}}

The type of {\tt{y}} is one of scalar, structure or union.

\section{{\tt{goto label}}, {\tt{if y op z then goto label}}, {\tt{label:}}}
\label{_3ac_e002}
These represents unconditional jump, conditional jump and
jump address,  respectively. The structure according to these are
like below.
\begin{verbatim}
struct goto3ac : tac {
  to3ac* m_to;  // contain jump address
  enum op { NONE, EQ, NE, LE, GE, LT, GT };
  op m_op;
  goto3ac() : tac(0,0,0), m_to(0), m_op(NONE) {}
  goto3ac(op op, var* y, var* z) : tac(0,y,z), m_to(0), m_op(op) {}
};

struct to3ac : tac {
  vector<goto3ac*> m_goto;  // contain referencing jumps
  to3ac() : tac(0,0,0) {}
};
\end{verbatim}
Unconditional jump are treated as special case of conditional jump.
Conditional jump operators (above {\tt{enum op}} member) 
are the same meaning with 
{\tt{==}}, {\tt{!=}}, {\tt{<=}}, {\tt{>=}}, {\tt{<}}, {\tt{>}}
in C language.
Frontend must generates these 3 address codes of which operands make sense.
Backend can distingish various {\tt{if y $op$ z then goto label}}
from the type of {\tt{y}}.

\begin{QandA}
Why is jump address implemented as 3 address code?

Answer : In optimization phase, we'll want to invesigate
the infomation described in above structure. So, for example,
{\tt{string}} is not enough to express jump address.
\end{QandA}

\section{ {\tt{x := va\_start y}}, {\tt{x := va\_arg y, type}}, {\tt{va\_end3ac}}}

In some implementations of backend,
these are implemented as C preprocessor macro.
But in other implementations, it's difficult to implement just using
C preprocessor macro. If backend can generate target code
for these 3 address code, it is simple.

\section{alloca x, y}
\label{_3ac_e003}
Variable length array in specification ISO C. 
\begin{verbatim}
void f(int n)
{
  int a[n];  /* variable length arrray */
  ...
}
\end{verbatim}
3 address code for this program is like below
\begin{verbatim}
f:
    t0 := 4 * n    # assume sizeof(int) = 4
    alloca a, t0   # assign t0 bytes memory for `a'
    ...
\end{verbatim}

\section{asm ``string''}

{\tt{asm-declaration}} is in ISO C++ but not in C.
Not necessary but we can change grammer.


